\documentclass{article}
\usepackage{graphicx} % Required for inserting images
\usepackage{amssymb}
\usepackage{amsthm}
\usepackage{amsmath}
\usepackage{cleveref}
\usepackage{natbib}

\newtheorem{theorem}{Theorem}[section]
\newtheorem{question}{Question}
\newtheorem{proposition}[theorem]{Proposition}
\newtheorem{observation}[theorem]{Observation}
\newtheorem{lemma}[theorem]{Lemma}
\newtheorem{corollary}[theorem]{Corollary}
\newtheorem{conjecture}[theorem]{Conjecture}
\newtheorem{definitions}[theorem]{Definitions}
\newtheorem{definition}[theorem]{Definition}
\newtheorem{exercise}[theorem]{Exercise}
\newtheorem{exercises}[theorem]{Exercise}

\newtheorem{example}[theorem]{Example}
\newtheorem{examples}[theorem]{Examples}
\newtheorem{moreexamples}[theorem]{More Examples}
\newtheorem{remark}[theorem]{Remark}

\newtheorem{adventure}[theorem]{Adventure}

\newtheorem{fact}[theorem]{Fact}
\newtheorem{facts}[theorem]{Facts}

% complexity classes
\newcommand{\NC}{\mathsf{NC}}
\newcommand{\TC}{\mathsf{AC}}
\newcommand{\AC}{\mathsf{TC}}
\newcommand{\poly}{\mathsf{poly}}
\newcommand{\logspace}{\mathsf{L}}


% transformers
\newcommand{\COT}{\ensuremath{\mathsf{COT}}}

\title{Chain of Thought Notes}
\author{ayang4 }
\date{August 2025}

\begin{document}

\maketitle

\section{Initial Thoughts}

\begin{itemize}
    \item Started thinking about enumeration classes using CoT. Bounding the context window is like bounding the amount of information given for the next step of enumeration
    \item We should allow a pointer to a position, so the machine can know where to retrieve the bounded amount of data from
    \item Question: is there a problem where a CoT solution needs access to all of the previous outputs at each step?
    \item Question: can we separate CoT with context size $k$ from context size $k+1$? 
\end{itemize}

\section{COT with Bounded Context Windows}

\begin{definition}
    We say $\COT[t(n),s(n)]$ is the class of problems solvable by transformers using COT with $t(n)$ steps and $s(n)$ context window size.
\end{definition}

The main claim at the end was that 

\[\NC^1\subseteq \COT[\poly(n),k] \subseteq \COT[\poly(n),\log(n)]\subseteq \logspace\]


\subsection{Log Size}

Since one forward-pass of a transformer is in $\TC^0\subseteq\logspace$ and we can also store an additional pointer in $\log(n)$ bits, we can have $\COT$ with $\poly(n)$ steps still take $\log$ space. 

\subsection{Constant Size}

With $\COT$ we can simulate $\NC^1$, like simulating an automaton transitioning from state to state at each time step. 
The other more ambitious goal was to show $\COT[\poly(n),k] \subseteq \NC^1$ by simulation of $\COT$ transformers using BWBPs. 

The main idea is to simulate $t(n)$ steps of $\COT$ using a program with $t(n)$ constant-size blocks each simulating a $\TC^0$ circuit capturing the transformer, and forwarding $k$ bits of information to the next block. 
It is not clear whether this forwarding of $k$ bits will still be simulatable by a BWBP.

\subsection{Separations}

The idea is then to separate $\COT[\poly(n),k] \subseteq \COT[\poly(n),\log(n)]$.

\section{Other Stuff}

I recently saw this paper \citep{li2025constant} which looked at COT with constant context window size and showed Turing-completeness. 
I think their construction is extremely non-uniform (they have a positional encoding $\mathsf{pe}(i)$ which at every step $i$ basically computes the position that the Post machine's head should attend to. 
Not sure if it will be relevant but seems good to know about this.

\bibliographystyle{plainnat}
\bibliography{references}

\end{document}
